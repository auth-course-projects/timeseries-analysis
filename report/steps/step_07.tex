\chapter{Συμπερασματικά Σχόλια}
\label{ch:step7}
\thispagestyle{fancy}

Συγκρίνοντας τα αποτελέσμστα και σημεία αλλαγής με αυτά του αντίστοιχου \tl{paper} οι ακόλουθες παρατηρήσεις προκύπτουν:
\begin{itemize}
    \item Η χρησιμοποίηση \textbf{γραμμικών} μοντέλων πρόβλεψης των στάσιμων χρονοσειρών (που έγιναν στάσιμες μετά από τις αντίστοιχες ενέργειες που περιγράφηκαν στα βήματα \ref{ch:step1} και \ref{ch:step3}) οδηγεί εν γένει σε \textbf{καλύτερα αποτελέσματα} τόσο σε ότι αφορά τα σφάλματα πρόβλεψης όσο και στα σημεία αλλαγής που επιλέγονται από τη μέθοδο, τα οποία είναι σε σχετική συμφωνία και με τις θέσεις που επιλέγονται απο τη μέθοδο του προτεινόμενου \tl{paper}
    \item Η χρησιμοποίηση \textbf{μη-γραμμικών} τοπικών μοντέλων μέσου όρου αν και κάποιες φορές τείνει να δείνει τα ίδια ή κοντινά σημεία αλλαγής, δεν το επιτυγχάνει πάντα και δεδομένου των μεγαλύτερων σφαλμάτων πρόβλεψης τη θεωρούμε \textbf{υποδεέστερη}, κάποιες φορές μόλις καλύτερη από τη πρόβλεψη της μέσης τιμής
    \item Η πρόβλεψη με \textbf{γραμμικά} μοντέλα αν και καλύτερη από τα μη-γραμμικά τοπικά μοντέλα μέσου όρου, είναι \textbf{σημαντικά πιο αργή} στην εκτέλεσή της κάτι που οφείλεται κυρίως στο ότι πρέπει να εκτιμηθουν οι παράμετροι των εκάστοτε $ARMA$ μοντέλων
\end{itemize}

\textit{Γενικότερα, είμαι πεπεισμένος πως αυτός ο τρόπος επιλογής σημείων αλλαγής αν και όχι τόσο πολύπλοκος θεωρητικά και αλγοριθμικά, είναι ικανός να δώσει σωστά σημεία αλλαγής με αυτόματο και ευσταθή τρόπο. Πιθανή βελτίωση θα μπορούσε να αποτελέσει η δοκιμή και άλλων στατιστικών των σφαλμάτων πρόβλεψης, όπως το \tl{MSE}. Επίσης, και με δεδομένο την ανωτερότητα των γραμμικών μοντέλων στη συγκεκριμένη μέθοδο, θα ήταν σκόπιμο να δοκιμαστεί η πρόβλεψη και με τοπικά γραμμικά μοντέλα πρόβλεψης (\tl{local liner prediction models}) στη θέση των τοπικών μοντέλων πρόβλεψης μέσου όρου (\tl{local average prediction models}).}