\chapter{Προσαρμογή Γραμμικού Μοντέλου}
\label{ch:step2}
\thispagestyle{fancy}

Με βάση τα συμπερασμάτα που εξήχθηκαν στην ανάλυση του πρώτου βήματος το γραμμικό μοντέλο που θα προσαρμόσουμε στη χρονοσειρά Α είναι \tl{ARIMA} με $d=1$, καθώς:
\begin{itemize}
    \item παίρνοντας τις πρώτες διαφορές (ενν. στη μετασχηματισμένη με τετραγωνικές ρίζες αρχική χρονοσειρά), η χρονοσειρά που προκύπτει φαίνεται στα διαγράμματα ιστορίας και αυτοσυσχέτισης να είναι στάσιμη (η τάση έχει απαλειφθεί) και επιπλέον δεν υπάρχει εξάρτηση της διασποράς από τη τάση
    \item δεν φαίνεται να υπάρχει εποχικότητα και άρα η χρήση ενός ARIMA μοντέλου θα είναι ικανοποιητική. Συγκεκριμένα δοκιμάστηκε η εκτίμηση της εποχικότητας (με μέσους όρους στοιχείων της περιόδου) και οι ακολουθίες που βρέθηκαν για περιόδους 5, 6 και 16 ημερών (καθώς αυτές είναι οι στατιστικά σημαντικές τιμές του διαγράμματος αυτοσυσχέτισης - σχήμα \ref{fig:xa_sqrt_of_1st_differences_autocorrelation}) είναι [0.082,-0.0672,0.021,0.039,-0.075], [0.011,-0.033,-0.017,0.020,-0.106,0.1250] και [0.087,0.224,-0.176,-0.104,0.092,0.214,-0.057,-0.130,0.058,-0.095,0.020,0.046,-0.069,0.219,-0.254,-0.075] αντίστοιχα. Δεδομένου ότι το εύρος τιμών της χρονοσειράς των πρώτων διαφορών των τετραγωνικών ριζών είναι περίπου στο [-3, 3] αλλά και ότι οι τιμές της αυτοσυσχετίσεις για τις συγκεκριμένες υστερήσεις είναι οριακά πάνω από το όριο σημαντικότητας, μπορούμε με ασφάλεια να εξάγουμε το συμπέρασμα ότι δεν υπάρχει εποχικός όρος στη χρονοσειρά των πρώτων διαφορών των τετραγωνικών ριζών.
\end{itemize}

Στο σημείο αυτό κρίνεται σκόπιμο να τονισθεί για μία ακόμη φορά ότι η χρονοσειρά της οποίας θα παρθούν οι πρώτες διαφορές για να προσαρμοστεί μοντέλο $ARIMA(p,1,q)$ δεν είναι η αρχική χρονοσειρά, $\{Y_a(t)\}$, αλλά η χρονοσειρά των τετραγωνικών ριζών, $\{\sqrt{Y_a(t)}\}$ ώστε να έχουμε σταθεροποιήσει τη διασπορά (την εξάρτησή της από την τάση). Η ανάλυση που ακολουθεί, επομένως, αφορά τη χρονοσειρά των τετραγωνικών ριζών και παρατίθεται ακολούθως.

\section{Προσαρμογή \tl{ARIMA(p,1,q)}}

Έτσι, συνεχίζοντας την ανάλυση από το τέλος του βήματος 1, μετά την εφαρμογή των πρώτων διαφορών στη χρονοσειρά των τετραγωνικών ριζών έχουμε υλοποιήσει το πρώτο στάδιο προσαρμογής ενός μοντέλου \tl{ARIMA(p,d,q)} που είναι η εφαρμογή διαφορών \tl{d}-οστής. Άρα εδώ θα είναι $d=1$ και η χρονοσειρά που καταλήγουμε είναι η $\{X_a(t)\}$.\\Το δεύτερο στάδιο είναι η εύρεση των παραμέτρων του μοντέλου \tl{ARMA(p,q)}, κάτι που αναλύεται στην επόμενη υπο-ενότητα. 

\subsection{Προσαρμογή \tl{ARMA(p,q)} στη χρονοσειρά πρώτων διαφορών}

Οι παράμετροι \tl{p} και \tl{q} είναι \tl{hyperparameters} του μοντέλου και επομένως το πρώτο μας μέλημα είναι να βρούμε το βέλτιστο συνδυσμό την παραμέτρων αυτών ή τάξεων του μοντέλου. Προς το σκοπό αυτό θα κάνουμε \tl{grid search} για τιμές των τάξεων από 0 (απουσία του αντίστοιχου  όρου) έως και 10.\par

Για την αξιολόγηση του κάθε συνδυασμού χρησιμοποιήθηκαν τα κριτήρια πληροφορίας \tl{Akaike (AIC)} και \tl{Forward Prediction Error (FPE)}, τα οποία ορίζονται ως εξής:
\begin{align}
AIC(p,q) = \ln{\left(s_z^2\right)} + \frac{2\times(p+q)}{n} 
\end{align}
και
\begin{align}
FPE(p,q) = s_z^2 \times \frac{n + (p+q)}{n - (p+q)}
\end{align}
όπου $n$ είναι ο αριθμός των δειγμάτων που χρησιμοποιήθηκαν για την εκτίμηση των ροπών (π.χ για τη δειγματική αυτοσυσχέτιση) και $s_z^2$ είναι η (δειγματική) διασπορά των σφαλμάτων ή υπολοίπων που προκύπτουν όταν συγκρίνουμε τις τιμές του προσαρμοεσμένου μοντέλου τάξης $(p,q)$ με τις πραγματικές τιμές της χρονοσειράς (ενν. τη χρονοσειρά $\{X_a(t)\}$ που προέκυψε ως οι πρώτες διαφορές της χρονοσειράς των τετραγωνικώ ριζών της αρχικής χρονοσειράς προβολών του βίντεο Α, $\{Y_a(t)\}$). \par

Παρακάτω παρατίθενται οι τιμές του \tl{AIC} για τους συνδυασμούς των παραμέτρων $(p,q)=(0...10,0...10)$, φυσικά με εξαίρεση το συνδυασμό $(p,q)=(0,0)$:

\begin{table}[H]
\resizebox{\textwidth}{!}{%
\centering
\begin{tabular}{ |c|c|c|c|c|c|c|c|c|c|c|c|c| }
\hline
\space&\multicolumn{12}{c|}{$p$} \\\hline
\multirow{2}{*}{$q$} & \space & 0 & \cellcolor[HTML]{EDEDED}1 & 2 & 3 & 4 & 5 & 6 & 7 & 8 & 9 & 10\\\cline{2-13}
& \cellcolor[HTML]{EDEDED} 0&\space& \cellcolor{lightgray} -0.465&-0.463&-0.462&-0.461&-0.460&-0.459&-0.459&-0.457&-0.458&-0.457\\\cline{2-13}
&1&-0.211&-0.463&-0.463&-0.462&-0.462&-0.461&-0.460&-0.458&-0.456&-0.457&-0.456\\\cline{2-13}
&2&-0.306&-0.462&-0.462&-0.465&-0.463&-0.461&-0.460&-0.457&-0.456&-0.459&-0.458\\\cline{2-13}
&3&-0.376&-0.461&-0.462&-0.463&-0.462&&-0.458&-0.459&-0.457&-0.458&-0.457\\\cline{2-13}
&4&-0.395&-0.46&-0.461&-0.462&&&-0.463&-0.460&-0.462&-0.457&-0.456\\\cline{2-13}
&5&-0.427&-0.460&-0.459&-0.460&-0.459&-0.461&-0.459&&-0.455&-0.455&-0.457\\\cline{2-13}
&6&-0.436&-0.459&-0.458&-0.459&-0.459&-0.460&&-0.463&-0.457&-0.458&-0.455\\\cline{2-13}
&7&-0.448&-0.458&-0.456&-0.457&-0.457&-0.457&-0.456&-0.457&-0.462&-0.454&-0.453\\\cline{2-13}
&8&-0.449&-0.458&-0.456&-0.459&-0.458&-0.456&-0.456&-0.457&-0.454&-0.462&-0.455\\\cline{2-13}
&9&-0.447&-0.456&-0.456&-0.455&-0.456&-0.456&-0.455&-0.455&-0.454&-0.452&-0.456\\\cline{2-13}
&10&-0.446&-0.456&-0.454&-0.454&-0.455&-0.454&-0.453&-0.459&-0.456&&-0.456\\\hline
\end{tabular}}
\caption{Αναζήτηση Πλέγματος με βάση τη μετρική \tl{AIC} για διάφορες τιμές των τάξεων $(p,q)$}
\label{table:grid_search_aic}
\end{table}

από όπου φαίνεται πως η χαμηλότερη τιμή του \tl{AIC} επιτυγχανέται όταν προσαρμοζέται μοντέλο \tl{ARMA(0,1)} ή, ισοδύναμα, μοντέλο \tl{MA(1)}. Ακολούθως δίνονται οι τιμές του \tl{FPE} για τους αντίστοιχους συνδυασμούς τιμών των \tl{p} και \tl{q}:

\begin{table}[H]
\resizebox{\textwidth}{!}{%
\centering
\begin{tabular}{ |c|c|c|c|c|c|c|c|c|c|c|c|c| }
\hline
\space&\multicolumn{12}{c|}{$p$} \\\hline
\multirow{2}{*}{$q$} & \space & 0 & \cellcolor[HTML]{EDEDED}1 & 2 & 3 & 4 & 5 & 6 & 7 & 8 & 9 & 10\\\cline{2-13}
& \cellcolor[HTML]{EDEDED} 0&\space& \cellcolor{lightgray} 0.628 &0.629&0.630&0.631&0.632&0.632&0.632&0.633&0.632&0.633\\\cline{2-13}
&1&0.810&0.629&0.629&0.630&0.630&0.631&0.631&0.632&0.634&0.633&0.634\\\cline{2-13}
&2&0.737&0.630&0.630&0.628&0.629&0.630&0.631&0.633&0.634&0.632&0.632\\\cline{2-13}
&3&0.687&0.631&0.630&0.629&0.630&&0.633&0.632&0.633&0.632&0.633\\\cline{2-13}
&4&0.673&0.631&0.631&0.630&&&0.629&0.631&0.630&0.633&0.634\\\cline{2-13}
&5&0.652&0.632&0.632&0.631&0.632&0.631&0.632&&0.634&0.634&0.633\\\cline{2-13}
&6&0.647&0.632&0.633&0.632&0.632&0.631&&0.630&0.633&0.633&0.634\\\cline{2-13}
&7&0.639&0.633&0.634&0.633&0.633&0.633&0.634&0.633&0.630&0.635&0.635\\\cline{2-13}
&8&0.639&0.633&0.634&0.632&0.632&0.634&0.634&0.633&0.635&0.630&0.635\\\cline{2-13}
&9&0.640&0.634&0.634&0.635&0.634&0.634&0.634&0.634&0.635&0.636&0.634\\\cline{2-13}
&10&0.640&0.634&0.635&0.635&0.635&0.635&0.636&0.632&0.634&&0.634\\\hline
\end{tabular}}
\caption{Αναζήτηση Πλέγματος με βάση τη μετρική \tl{FPE} για διάφορες τιμές των τάξεων $(p,q)$}
\label{table:grid_search_fpe}
\end{table}

Όπως επιβεβαιώνεται και από τους δύο πίνακες παραπάνω, φαίνεται πως από τα γραμμικά μοντέλα καλύτερα προσαρμόζεται το μοντέλο $ΜΑ(1)$. Ωστόσο, θα θέλαμε το μοντέλο μας να έχει \textquote{μνήμη} κάτι που επιτυγχάνεται έαν έχει όρους ανάδρασης ή αυτοπαλινδρόμησης. Για το λόγο αυτό και παρατηρώντας και τους πίνακες με τις τιμές των κριτηρίων πληροφορίας παραπάνω, θα κάνουμε διάγνωση καταλληλότητας τόσο για το $MA(1)$ όσο και για το $AR(1,1)$ που φαίνεται να έχει πολύ κοντινές τιμές στα κριτήρια πληροφοίας.

\parΠριν προχωρήσουμε στη διάγνωση καταλληλότητας, θα ήταν σκόπιμο να τονιστεί ότι δεν χρησιμοποιήθηκε το \tl{NRMSE} (πρόβλεψης 1 βήματος μπροστά) ή το $s_z^2$ των υπολοίπων για επιλογή των τάξεων του μοντέλου καθώς αμφότερα δεν λαμβάνουν υπόψη τους την \textquote{πολυπλοκότητα} του μοντέλου και άρα θα μειώνονταν με αύξηση των τάξεων κάτι που οδηγεί σε πιθανό \tl{overfitting}. Πράγματι, με βάση το \tl{NRMSE} πρόβλεψης 1 βήματος ο καλύτερος συνδυασμός παραμέτρων θα ήταν $(p,q) = (10,10)$ με αντίστοιχο \tl{NRMSE} 0.744.

\subsection{Διάγνωση καταλληλότητας του μοντέλου MA(1)}
Το μοντέλο $MA(1)$ που εκτιμήθηκε από τη συνάρτηση \texttt{\tl{fitARMA}()} και προσαρμόζεται στη χρονοσειρά των διαφορών των τετρ. ριζών, $\{X_a(t) = \sqrt{Y_a(t)} - \sqrt{Y_a(t-1)}\}, t=2,...,1199$, είναι το εξής:
\begin{align}
x_t = -0.0033 + z_t - 0.8426 \  z_{t-1}, \ \ \  t=2,..,1199
\end{align}

όπου ο μέσος όρος της χρονοσειράς $\{X_a(t)\}$ είναι $\overline{x_a}=-0.0033$. Ενώ η εκτίμηση της τυπικής απόκλισης των σφαλμάτων ή υπολοίπων προσαρμογής βρέθηκε να είναι ίση με \textbf{$s_z = 0.7922$} (εκτίμηση διασποράς ίση με $s_z^2 = 0.6275$). Το αντίστοιχο μοντέλο για την αρχική χρονοσειρά των προβολών του βίντεο Α, θα είναι:
\begin{align}
Y_t = \left(\sqrt{Y_{t-1}} -0.0033 + Z_t - 0.8426 \ Z_{t-1} \right)^2, \ \ \  t=2,..,1199
\end{align}




\section{Τελικό Μοντέλο και Σφάλματα Προσαρμογής}

μπλα μπλα μπλα