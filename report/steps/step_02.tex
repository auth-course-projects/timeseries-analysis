\chapter{Προσαρμογή Γραμμικού Μοντέλου}
\label{ch:step2}
\thispagestyle{fancy}

Με βάση τα συμπερασμάτα που εξήχθηκαν στην ανάλυση του πρώτου βήματος το γραμμικό μοντέλο που θα προσαρμόσουμε στη χρονοσειρά Α είναι \tl{ARIMA} με $d=1$, καθώς θα χρησιμοποιηθούν διαφορές πρώτης τάξης. Η χρονοσειρά που θα προσαρμοστεί το μοντέλο αυτό δεν είναι η αρχική χρονοσειρά, $Y_a(t)$, αλλά η χρονοσειρά των τετραγωνικών ριζών, $\sqrt{Y_a(t)}$ ώστε να έχουμε σταθεροποιήσει τη διασπορά (συγκ. την εξάρτησή της από την τάση) πριν την προσαρμογή του γραμμικού μοντέλου. Η ανάλυση που ακολουθεί, επομένως, αφορά τη χρονοσειρά των τετραγωνικών ριζών και παρατίθεται ακολούθως.

\section{Προσαρμογή \tl{ARIMA(p,1,q)}}

Μετά την εφαρμογή των πρώτων διαφορών στη χρονοσειρά των τετραγωνικών ριζών έχουμε υλοποιήσει το πρώτο στάδιο προσαρμογής ενός μοντέλου \tl{ARIMA(p,1,q)} που είναι η εφαρμογή διαφορών \tl{p}-οστής τάξης. Το δεύτερο στάδιο είναι η εύρεση των παραμέτρων του μοντέλου \tl{ARMA(p,q)}, κάτι που αναλύεται στην επόμενη υπο-ενότητα. 

\subsection{Προσαρμογή \tl{ARMA(p,q)} στη χρονοσειρά πρώτων διαφορών}

Οι παράμετροι \tl{p} και \tl{q} είναι \tl{hyperparameters} του μοντέλου και επομένως το πρώτο μας μέλημα είναι να βρούμε το βέλτιστο συνδυσμό την τάξεων του μοντέλου. Προς το σκοπό αυτό θα κάνουμε \tl{grid search} για τιμές των τάξεων από 0 (απουσία του αντίστοιχου  όρου) έως και 10.\par

Για την αξιολόγηση του κάθε συνδυασμού χρησιμοποιήθηκαν τα κριτήρια πληροφορίας \tl{Akaike (AIC)} και \tl{Forward Prediction Error (FPE)}, τα οποία ορίζονται ως εξής:

\[ AIC(p,q) = \ln{(s_z^2)} + \frac{2\times(p+q)}{n} \]

και

\[ FPE(p,q) = s_z^2 \times \frac{n + (p+q)}{n - (p+q)} \]

όπου $n$ είναι ο αριθμός των δειγμάτων που χρησιμοποιήθηκαν για την εκτίμηση των ροπών (π.χ για τη δειγματική αυτοσυσχέτιση) και $s_z^2$ είναι η (δειγματική) διασπορά των σφαλμάτων ή υπολοίπων που προκύπτουν όταν συγκρίνουμε τις τιμές του προσαρμοεσμένου μοντέλου τάξης $(p,q)$ με τις πραγματικές τιμές της χρονοσειράς (ενν. τη χρονοσειρά $\{X_a(t)\}$ που προέκυψε ως οι πρώτες διαφορές της χρονοσειράς των τετραγωνικώ ριζών της αρχικής χρονοσειράς προβολών του βίντεο Α, $\{Y_a(t)\}$). \par

Παρακάτω παρατίθενται οι τιμές του \tl{AIC} για τους συνδυασμούς των παραμέτρων $(p,q)=(0...10,0...10)$, φυσικά με εξαίρεση το συνδυασμό $(p,q)=(0,0)$:

\begin{table}[h!]
\resizebox{\textwidth}{!}{%
\centering
\begin{tabular}{ |c|c|c|c|c|c|c|c|c|c|c|c|c| }
\hline
\space&\multicolumn{12}{c|}{$p$} \\\hline
\multirow{2}{*}{$q$} & \space & 0 & \cellcolor[HTML]{EDEDED}1 & 2 & 3 & 4 & 5 & 6 & 7 & 8 & 9 & 10\\\cline{2-13}
& \cellcolor[HTML]{EDEDED} 0&\space& \cellcolor{lightgray} -0.465&-0.463&-0.462&-0.461&-0.460&-0.459&-0.459&-0.457&-0.458&-0.457\\\cline{2-13}
&1&-0.211&-0.463&-0.463&-0.462&-0.462&-0.461&-0.460&-0.458&-0.456&-0.457&-0.456\\\cline{2-13}
&2&-0.306&-0.462&-0.462&-0.465&-0.463&-0.461&-0.460&-0.457&-0.456&-0.459&-0.458\\\cline{2-13}
&3&-0.376&-0.461&-0.462&-0.463&-0.462&&-0.458&-0.459&-0.457&-0.458&-0.457\\\cline{2-13}
&4&-0.395&-0.46&-0.461&-0.462&&&-0.463&-0.460&-0.462&-0.457&-0.456\\\cline{2-13}
&5&-0.427&-0.460&-0.459&-0.460&-0.459&-0.461&-0.459&&-0.455&-0.455&-0.457\\\cline{2-13}
&6&-0.436&-0.459&-0.458&-0.459&-0.459&-0.460&&-0.463&-0.457&-0.458&-0.455\\\cline{2-13}
&7&-0.448&-0.458&-0.456&-0.457&-0.457&-0.457&-0.456&-0.457&-0.462&-0.454&-0.453\\\cline{2-13}
&8&-0.449&-0.458&-0.456&-0.459&-0.458&-0.456&-0.456&-0.457&-0.454&-0.462&-0.455\\\cline{2-13}
&9&-0.447&-0.456&-0.456&-0.455&-0.456&-0.456&-0.455&-0.455&-0.454&-0.452&-0.456\\\cline{2-13}
&10&-0.446&-0.456&-0.454&-0.454&-0.455&-0.454&-0.453&-0.459&-0.456&&-0.456\\\hline
\end{tabular}}
\caption{Αναζήτηση Πλέγματος με βάση τη μετρική \tl{AIC} για διάφορες τιμές των τάξεων $(p,q)$}
\label{table:grid_search_aic}
\end{table}

από όπου φαίνεται πως η χαμηλότερη τιμή του \tl{AIC} επιτυγχανέται όταν προσαρμοζέται μοντέλο \tl{ARMA(0,1)} ή, ισοδύναμα, μοντέλο \tl{MA(1)}. Ακολούθως δίνονται οι τιμές του \tl{FPE} για τους αντίστοιχους συνδυασμούς τιμών των \tl{p} και \tl{q}:

\begin{table}[h!]
\resizebox{\textwidth}{!}{%
\centering
\begin{tabular}{ |c|c|c|c|c|c|c|c|c|c|c|c|c| }
\hline
\space&\multicolumn{12}{c|}{$p$} \\\hline
\multirow{2}{*}{$q$} & \space & 0 & \cellcolor[HTML]{EDEDED}1 & 2 & 3 & 4 & 5 & 6 & 7 & 8 & 9 & 10\\\cline{2-13}
& \cellcolor[HTML]{EDEDED} 0&\space& \cellcolor{lightgray} 0.628 &0.629&0.630&0.631&0.632&0.632&0.632&0.633&0.632&0.633\\\cline{2-13}
&1&0.810&0.629&0.629&0.630&0.630&0.631&0.631&0.632&0.634&0.633&0.634\\\cline{2-13}
&2&0.737&0.630&0.630&0.628&0.629&0.630&0.631&0.633&0.634&0.632&0.632\\\cline{2-13}
&3&0.687&0.631&0.630&0.629&0.630&&0.633&0.632&0.633&0.632&0.633\\\cline{2-13}
&4&0.673&0.631&0.631&0.630&&&0.629&0.631&0.630&0.633&0.634\\\cline{2-13}
&5&0.652&0.632&0.632&0.631&0.632&0.631&0.632&&0.634&0.634&0.633\\\cline{2-13}
&6&0.647&0.632&0.633&0.632&0.632&0.631&&0.630&0.633&0.633&0.634\\\cline{2-13}
&7&0.639&0.633&0.634&0.633&0.633&0.633&0.634&0.633&0.630&0.635&0.635\\\cline{2-13}
&8&0.639&0.633&0.634&0.632&0.632&0.634&0.634&0.633&0.635&0.630&0.635\\\cline{2-13}
&9&0.640&0.634&0.634&0.635&0.634&0.634&0.634&0.634&0.635&0.636&0.634\\\cline{2-13}
&10&0.640&0.634&0.635&0.635&0.635&0.635&0.636&0.632&0.634&&0.634\\\hline
\end{tabular}}
\caption{Αναζήτηση Πλέγματος με βάση τη μετρική \tl{FPE} για διάφορες τιμές των τάξεων $(p,q)$}
\label{table:grid_search_fpe}
\end{table}

Όπως επιβεβαιώνεται και από τους δύο πίνακες παραπάνω, φαίνεται πως από τα γραμμικά μοντέλα καλύτερα προσαρμόζεται το μοντέλο $ΜΑ(1)$.

\section{Μοντέλο και Σφάλματα Προσαρμογής}

μπλα μπλα μπλα